\documentclass[11pt]{article}
\newcommand{\thetitle}{HYPERION Memo \#1: The Relationship Between System 
Temperature and Sky Beam Coverage}
\newcommand{\theauthor}{Kara Kundert}
\newcommand{\theauthorsemail}{kkundert@berkeley.edu}
\newcommand{\thedate}{June 23, 2016}
% the following controls some aspects of how the text is displayed on the page
\setlength{\textwidth}{6.5in}
\setlength{\textheight}{8.25in}
\setlength{\oddsidemargin}{0in}

% set up the page headers and footers
\usepackage{fancyhdr}
    \pagestyle{fancy}
    \lhead{\sffamily\slshape\small\thetitle}
    \rhead{\sffamily\small\theauthor}
    \cfoot{\sffamily\slshape\small\thepage}

% support display of graphics
\usepackage{graphicx}

% the following control some aspects of how paragraphs are displayed
\parindent=0pt
\parskip=2ex

% import library of technical symbols
\usepackage{amsmath,amssymb,latexsym}

% import bibliography tools
\usepackage{natbib}
\citestyle{aa}

\begin{document}
% print the title in san-serif font, in bold, in huge characters
\title{
    \sffamily\bfseries\huge
    \thetitle \\
}
% print the author in san-serif font
\author{
    \sffamily\theauthor \\
    \sffamily\theauthorsemail
}
\date{\thedate}
\maketitle
\sloppy

\section{Introduction}

In this memo, we seek to explore in further detail the idea of an 
interferometric study of the spatial monopole of the 21cm brightness 
temperature as a function of redshift (i.e. the ``global signal"). In order to 
directly sample the zero-spacing mode of the sky, most previous studies of the 
global signal have been single-element experiments. However, it is our 
assertion that an interferometric approach can not only help to mitigate many 
of the systematics inherent to single-element experiments, but can also, given 
a clever experimental design and approach to data analysis, successfully sample 
the spatial monopole. 

One of the modifications we can make to our experimental design is to 
artificially impose a spatial scale on the monopole through the creation of a 
horizon. From a Fourier perspective, the existence of the horizon interrupts 
the flat nature of the spatial monopole, which creates leakage from the DC-mode 
into modes with non-zero interferometric spacings. As discussed in 
\cite{presley2015}, a spacing of approximately one wavelength is ideal in 
balancing engineering constraints while maximizing reception of the spatial 
monopole.

We note in this memo that, by manipulating the scale of the horizon, we are 
similarly able to manipulate this Fourier leakage and thereby observe the 
monopole signal from other Fourier modes.  By constructing absorptive 
beam-shaping baffles around the individual elements of our interferometer, we 
believe that we can optimize our proposed global signal interferometric 
experiment.

\section{Simulation}

The simulation calculates the system temperature as shown below in 
Eq.~\eqref{eq:sys-temp}, where $T_{sys}$ is the overall system temperature, 
$T_{rx}$ is the receiver noise temperature, $T_{abs}$ is the temperature of the 
absorptive baffle structures, $T_{sky}$ is the sky temperature, $T_{21}$ is the 
21cm monopole brightness temperature, $\Omega$ is the full visible sky coverage 
from horizon to horizon, and $\Omega'$ is the baffle-modified sky coverage.

\begin{equation}
    \label{eq:sys-temp}
    T_{sys} = \frac{T_{rx} \Omega + T_{abs} (\Omega - \Omega') + (T_{sky} + 
    T_{21}) \Omega'}{\Omega'}
\end{equation}

For the purposes of this test, we have set some of the above parameters to be 
constants. The baffle absorber was set to be a standard blackbody with a 
temperature $T_{abs} = 300$ K. The overall sky temperature $T_{sky}$ was 
calculated off of numbers presented in~\cite{rogers2008} and models presented 
in~\cite{haslam1982}.  With the initial brightness temperature measured to be 
$T = 237$ K at $\nu = 150$ MHz, Eq.~\eqref{eq:synch-temp} was implemented with 
a final frequency of $\nu = 70$ MHz and a spectral index $\beta = 2.5$ to 
receive a final galactic brightness temperature of approximately $T_{sky} = 
1600$ K.  Finally, the best estimate of the brightness temperature of the 
global sky signal is approximately $T_{21} = 20$ mK~\citep{pritchard-loeb2010}.  
We have iterated over an initial selection of $T_{rx} = 50 - 300$ K, as we 
believe these to be potentially attainable noise levels for a non-cryogenically 
cooled low-noise amplifier. 

\begin{equation}
    \label{eq:synch-temp}
    T(\nu) = T(\nu_{150}) \Big(\frac{\nu}{\nu_{150}} \Big)^{-\beta}
\end{equation}

In this simulation, we aim to determine the regime over which the 
baffle-limited beam will enable Fourier spreading of the global signal into 
neighboring spatial frequency bins while simultaneously ensuring that the sky 
remains the dominant term in the system temperature, as that will give us the 
best chances later of recovering the extremely faint 21cm global sky signal.

\section{Conclusions}

\begin{figure}
    \begin{center}
    \includegraphics[width=\linewidth]{/home/kara/capo/kmk/scripts/sysTemp.png}
    \end{center}
    \caption{
        Pictured above is the relationship between the fractional sky coverage 
        as determined by the beam-limiting baffles and the overall system 
        temperature. It is readily apparent that below $10\%$ sky coverage, the 
        system temperature is completely dominated by the presence of the 300 K 
        baffles.
    }
    \label{fig:sys-temp}
\end{figure}

Pictured in Fig.~\ref{fig:sys-temp}, in a test with a fractional sky coverage 
$\Omega'/\Omega$ ranging from 0.01 to 1, it is clear that the vast majority of 
the parameter space has a system temperature dominated by the absorptive 
baffles. From this plot, we can conclude that -- at an absolute minimum -- we 
must construct our absorptive baffles to admit at least $10\%$ of the sky in 
order to ensure that the sky is the dominant term in our system. However, in 
order to ensure a system temperature of less than 6000 K, we will need to admit 
at least $20\%$ of the sky. We can also conclude that the receiver noise 
temperature is a relatively minor contributing factor to the overall system 
temperature, and that the experiment will not be strongly affected by the 
selection of cheaper 150 K amplifiers over more costly, lower noise options.

\bibliography{hyperion}{}
\bibliographystyle{apj}

\end{document}





